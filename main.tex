\documentclass{article}
\usepackage[utf8]{inputenc}
\usepackage{enumitem}

\begin{document}

\title{Survey acerca de sistemas de estacionamiento inteligentes}
\author{Geraldo Mark Romero Guillen}
\date{\today}

\maketitle

\section*{Abstract}
Recientemente el mundo ha visto grandes y geniales proyectos de automatización,
sin embargo el estacionamiento de vehículos sigue siendo un problema en muchas
ciudades del mundo. En este artículo se discutirá el estado actual de los
sistemas de estacionamiento inteligentes, sus ventajas, desventajas, y se
presentarán algunos casos de estudio. Finalmente, se discutirán las
posibilidades de desarrollo futuro en el campo de los sistemas de estacionamiento
inteligentes.

\section*{Introduction}
En los últimos cuatro años el aumento constante de la población y el crecimiento económico global ha impulsado
la adquisición masiva de automóviles en todo el mundo. Como resultado, las calles y
carreteras se han vuelto cada vez más congestionadas, lo que ha llevado a problemas
de tráfico con espacios de estacionamiento con una pésima gestión, esto impacta
directamente en la vida de las personas generando problemas como: restrasos y pérdida
de tiempo,  estrés y ansiedad. Este fenómeno destaca la necesidad
urgente de soluciones de estacionamiento autosostenible para abordar estos desafios.
Los sistemas de estacionamiento inteligentes son una solución a este problema, sin 
embargo los sistemas actuales presentan varios problemas detallados a continuación.
% Lista de problemas
\begin{itemize}
\item El pago por el servicio de estacionamiento es manual.
\item Largas colas para pagar el servicio de estacionamiento.
\item Inseguridad en los lotes de estacionamiento.
\item Falta de información sobre el estado de los lotes de estacionamiento.
\item Falta de información sobre la ubicación de los lotes de estacionamiento.
\item Costos de servicio poco claros y variables.
\item Falta de un mapa de los lotes de estacionamiento.
\end{itemize}
El objetivo principal de estas tecnologias es resolver los problemas mencionados,
adem\'as de reducir la carga y el estr\'es de los conductores.
Los sistemas de estacionamiento que funcionan con monedas son ineficientes y
poco seguros, ya que los conductores tienen que llevar consigo monedas para
poder pagar el servicio de estacionamiento, y los lotes de estacionamiento
tienen que tener personal para recibir el pago. Los sistemas de estacionamiento
que funcionan con tarjetas de prepago son un poco mejores, pero siguen siendo
ineficientes y poco seguros, ya que los conductores tienen que llevar consigo
una tarjeta de prepago para poder pagar el servicio de estacionamiento, y los
lotes de estacionamiento tienen que tener personal para recargar las tarjetas.
Los sistemas de estacionamiento inteligentes son una solución a estos problemas,
ya que permiten a los conductores pagar el servicio de estacionamiento de forma
automática, y permiten a los lotes de estacionamiento operar sin personal.


\section*{Background}
% Provide some background information on car parking systems and the need for intelligent systems.

\section*{Technologies Used in Intelligent Car Parking Systems}
% Discuss the various technologies used in Intelligent Car Parking Systems.

\section*{Advantages and Disadvantages of Intelligent Car Parking Systems}
% Discuss the pros and cons of Intelligent Car Parking Systems.

\section*{Case Studies}
% Discuss some case studies of Intelligent Car Parking Systems.

\section*{Future Directions}
% Discuss potential future developments in the field of Intelligent Car Parking Systems.

\section*{Conclusion}
% Conclude the paper.

\end{document}